\documentclass[11pt,
  paper=a4, 
  bibliography=totocnumbered,
	captions=tableheading,
	BCOR=10mm
]{scrreprt}

\usepackage[utf8]{inputenc}
 
 
\usepackage[onehalfspacing]{setspace}
\usepackage{csquotes} % Context sensitive quotation.
\usepackage{amsmath} % Standard math.
\usepackage{amsthm} % Math theorems.
\usepackage{amssymb} % More math symbols.
\theoremstyle{definition}
\newtheorem{definition}{Definition}[chapter]
 
\usepackage[section]{placeins} % Keep floats in the section they were defined in.
\usepackage{tabularx}
\usepackage{booktabs} % Scientific table styling.
\usepackage{floatrow} % Option for keeping floats in the place they were defined in the code.
\floatsetup[table]{style=plaintop}
\usepackage{hyperref} % Hyperlinks.
\usepackage[all]{nowidow} % Prevent widows and orphans.
\usepackage{xstring} % logic string operations
\usepackage[nopostdot, nonumberlist]{glossaries} % glossary for definitions and acronyms, without dot after entry and page reference 
\usepackage{bbm} % \mathbb on numerals.
\usepackage{csquotes}
\usepackage{mathtools}
\usepackage[ruled,vlined]{algorithm2e} % Pseudocode
\usepackage{scrhack} % Make warning go away.
\usepackage{graphicx}
\usepackage{subcaption} % Subfigures with subcaptions.
\usepackage{authoraftertitle} % Make author, etc., available after \maketitle
\usepackage{listofitems}
\usepackage{blindtext} % Placeholder text.
\usepackage[nopostdot, nonumberlist]{glossaries}
\makeglossaries % Generate the glossary

% \PassOptionsToPackage{obeyspaces}{url}%
\usepackage[backend=bibtex,% 
style=nature,% 
doi=true,isbn=false,url=false, eprint=false]{biblatex}
% \renewbibmacro*{url}{\printfield{urlraw}}

\addbibresource{references.bib}

\DeclareStyleSourcemap{
  \maps[datatype=bibtex, overwrite=true]{
    \map{
      \step[fieldsource=url, final]
      \step[typesource=misc, typetarget=online]
    }
    \map{
      \step[typesource=misc, typetarget=patent, final]
      \step[fieldsource=institution, final]
      \step[fieldset=holder, origfieldval]
    }
  }
}

%\linespread{1.5} % set line spacing
 
\usepackage{listings} % rendering program code
\lstset{% general command to set parameter(s)
	basicstyle=\ttfamily\color{grey},          % print whole listing small
	keywordstyle=\color{black}\bfseries\underbar,
	% underlined bold black keywords
	identifierstyle=,           % nothing happens
	commentstyle=\color{white}, % white comments
	stringstyle=\ttfamily,      % typewriter type for strings
	showstringspaces=false}     % no special string spaces


\DeclareFontFamily{U}{mathx}{\hyphenchar\font45}
\DeclareFontShape{U}{mathx}{m}{n}{
      <5> <6> <7> <8> <9> <10>
      <10.95> <12> <14.4> <17.28> <20.74> <24.88>
      mathx10
      }{}
\DeclareSymbolFont{mathx}{U}{mathx}{m}{n}
\DeclareFontSubstitution{U}{mathx}{m}{n}
\DeclareMathSymbol{\bigtimes}{1}{mathx}{"91}

 

%%% Custom definitions %%%
% Shorthands
\newcommand{\ie}{i.\,e.~}
\newcommand{\eg}{e.\,g.~}
\newcommand{\ind}{\mathbbm{1}}
% Functions
\newcommand{\tpow}[1]{\cdot 10^{#1}}
\newcommand{\figref}[1]{(Figure \ref{#1})}
\newcommand{\figureref}[1]{Figure \ref{#1}}
\newcommand{\tabref}[1]{(Table \ref{#1})}
\newcommand{\tableref}[1]{Table \ref{#1}}
\newcommand{\secref}[1]{%
	\IfBeginWith{#1}{chap:}{%
		(cf. Chapter \ref{#1})}%
		{(cf. Section \ref{#1})}%
		}
\newcommand{\sectionref}[1]{%
	\IfBeginWith{#1}{chap:}{%
		Chapter \ref{#1}}%
		{\IfBeginWith{#1}{s}{%
			Section \ref{#1}}%
			{[\PackageError{sectionref}{Undefined option to sectionref: #1}{}]}}}
\newcommand{\chapref}[1]{(see chapter \ref{#1})}
\newcommand{\unit}[1]{\,\mathrm{#1}}
\newcommand{\unitfrac}[2]{\,\mathrm{\frac{#1}{#2}}}
\newcommand{\codeil}[1]{\lstinline{#1}}{} % wrapper for preventing syntax highlight error
\newcommand{\techil}[1]{\texttt{#1}}
\newcommand{\Set}[2]{%
  \{\, #1 \mid #2 \, \}%
}
% Line for signature.
\newcommand{\namesigdate}[1][5cm]{%
	\vspace{5cm}
	{\setlength{\parindent}{0cm}
	\begin{minipage}{0.3\textwidth}
		\hrule 
		\vspace{0.5cm}
		{\small city, date}
	\end{minipage}
	 \hfill
	\begin{minipage}{0.3\textwidth}
		\hrule
		\vspace{0.5cm}
	    {\small signature}
	\end{minipage}
	}
}
% Automatically use the first sentence in a caption as the short caption.
\newcommand\slcaption[1]{\setsepchar{.}\readlist*\pdots{#1}\caption[{\pdots[1].}]{#1}}

% Variables. 
% Adapt if necessary, use to refer to figures and graphics.
\def \figwidth {0.9\linewidth}
\graphicspath{ {./graphics/figures/}{./graphics/figures/} } % Path to figures and images.


% Customizations of existing commands.
\renewcommand{\vec}[1]{\mathbf{#1}}
% Capitalized \autoref names.
\renewcommand*{\chapterautorefname}{Chapter}
\renewcommand*{\sectionautorefname}{Section}


% TODO Fill with your data.
\title{My full title}
\author{Firstname Lastname}

\begin{document}

\begin{titlepage}
	\begin{flushleft}
		Universität Osnabrück\\
		Fachbereich Humanwissenschaften\\
		Institute of Cognitive Science
	\end{flushleft}

	\vspace{2cm}
	\centering{
		Bachelor/Masterthesis\vspace{1cm}\\
		\textbf{\Large{\MyTitle}}
		\vspace{1cm}\\
		\begin{tabular}{c}
			\MyAuthor                          \\
			968684                             \\
			Bachelor's/Master's Program Cognitive Science \\
			Starting month and year - end month and year
		\end{tabular}}
	\vspace{1cm}

	\begin{tabular}{ll}
		First supervisor:  & Rüdiger Busche, M.Sc.          \\
		                   & inserve GmbH                   \\
		                   & Hannover                       \\\\
		Second supervisor: & Prof. Dr. Someone Else         \\
		                   & Institute of Cognitive Science \\
		                   & Osnabrück
	\end{tabular}

\end{titlepage}


\chapter*{Declaration of Authorship}
I hereby certify that the work presented here is, to the best of my knowledge and belief, original and the result of my own investigations, except as acknowledged, and has not been submitted, either in part or whole, for a degree at this or any other university.

\namesigdate
\pagenumbering{gobble}
\pagebreak

\begin{abstract}
	\textbf{\LARGE{Abstract}}\\\\
	%TODO summarize the main objectives and outcomes of your work. The abstract should fit on one page.
\end{abstract}




\tableofcontents
\listoffigures
\listoftables
\listofalgorithms


\chapter{Introduction}
\pagenumbering{arabic}
\blindtext

\begin{figure}[H]
	\centering
	\includegraphics[width=\figwidth]{scientific_paper_graph_quality}
	\slcaption{
		Developmemt of scientific paper graph quality. A dip in the
		quality of scientific graphs is observed from the early 1990s to the early 2010s.
		During this time Microsoft Paint and PowerPoint were often used to create graphs in scientific papers.\label{fig:scientific_graph_quality}}
\end{figure}

\begin{table}[H]
	\begin{tabular}{@{}ll@{}}
		\toprule
		year & quality \\ \midrule
		1985 & good    \\
		2000 & bad     \\ \midrule
		2015 & better  \\ \bottomrule
	\end{tabular}
	\caption{
		Empirical measurements of scientific graph quality. Data points were collected using
		a systematic literature review.\label{tab:scientific_graph_quality}}
\end{table}

This references a \figref{fig:scientific_graph_quality} while this references a table \tabref{tab:scientific_graph_quality}.

A citation looks like this \cite{hadash2018estimate}. To embed a citation in the text flow use textcite,
\eg \textcite{hadash2018estimate} said you should use a lot of citations.

\chapter{Methods}



\chapter{Results and Discussion}


\chapter{Conclusion}


\chapter*{Acknowledgements}
%TODO A place to say thank you to everybody who helped you.


% Acronym definitions
%TODO Add acronym definitions produced by acronyms2glossary.py 




\glsaddall
\printglossaries

\printbibliography

\end{document}